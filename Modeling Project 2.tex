\documentclass[12pt]{article}

\usepackage{amsfonts,latexsym,amsthm,amssymb,amsmath,amscd,euscript,mathrsfs}
\usepackage{framed}
\usepackage{fullpage}
\usepackage [english]{babel}
\usepackage [autostyle, english = american]{csquotes}
\usepackage{verbatim}
 \newcommand{\qpartial}[2]{\dfrac{\partial #1}{\partial #2}}



\title{Poetry and Mathematics in Skillistan} 
\author{Forrest Flesher, Vinh-Kha Le}

\begin{document}
\maketitle


\vspace*{0.15in}


Skillistan is a simple economy with only one type of firm that requires two types of workers:
those who can do math (M-types) and those who can write poems about math in addition
to doing math (P-types). The P-types are more productive than M-types. Firms
require some capital investment in the form of toys for the workers to play with while they
work. Residents can always choose to not work and earn a reservation wage pondering the
nature of reality. Note that the residents of Skillistan only derive utility from the wage
they obtain and none from leisure: pondering the nature of reality is no more pleasant
than doing math or writing poetry.

\begin{enumerate} 
\item Write down a model for the firm 
\begin{enumerate}
\item How many M-types and P-types will the firm hire?
\item What will be the wage offered to the M-types and the P-types?
\end{enumerate}
\end{enumerate}


\end{document}
