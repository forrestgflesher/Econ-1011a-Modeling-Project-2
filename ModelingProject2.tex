\documentclass[12pt]{article}

\usepackage{amsfonts,latexsym,amsthm,amssymb,amsmath,amscd,euscript,mathrsfs}
\usepackage{framed}
\usepackage{fullpage}
\usepackage [english]{babel}
\usepackage [autostyle, english = american]{csquotes}
\usepackage{verbatim}
 \newcommand{\qpartial}[2]{\dfrac{\partial #1}{\partial #2}}



\title{Poetry and Mathematics in Skillistan} 
\author{Forrest Flesher, Vinh-Kha Le}

\begin{document}
\maketitle


\vspace*{0.15in}


Skillistan is a simple economy with only one type of firm that requires two types of workers:
those who can do math (M-types) and those who can write poems about math in addition
to doing math (P-types). The P-types are more productive than M-types. Firms
require some capital investment in the form of toys for the workers to play with while they
work. Residents can always choose to not work and earn a reservation wage pondering the
nature of reality. Note that the residents of Skillistan only derive utility from the wage
they obtain and none from leisure: pondering the nature of reality is no more pleasant
than doing math or writing poetry.

\begin{enumerate} 
\item{Write down a model for the firm}
\begin{enumerate}
\item{How many M-types and P-types will the firm hire?}
\item{What will be the wage offered to the M-types and the P-types?}
\end{enumerate}

PART 1 HERE


\item{All residents graduate from Skillistan High where they learn how to do math without
paying any fees. When they graduate from high school, all residents of Skillistan are
identical M-types. Afterwards, they can choose to attend Skillvard College where
they can learn how to write poems about math and become P-types but they have
to pay tuition fees.}
\begin{enumerate}
\item{Derive a condition in terms of the wages in equilibrium for M-types and P-types
and in terms of the cost of Skillvard high for when M-types will choose to become
P-types.}
\end{enumerate}

PART 2 HERE

\item{Making any simplifying assumptions you may need to make:}
\begin{enumerate}
\item{Derive the labor demand curve}
\item{Derive the labor supply curve}
\item{Define the `poetry premium’ as the ratio of a P-type’s wage and an M-type’s
wage. Plot the labor demand and supply curves as a function of the number of
P-types in society and the poetry premium (in one graph).}
\item{Skillvard College announces an increase in tuition fees. What does this mean
for the supply of P-types and the wages they earn in equilibrium?}
\end{enumerate}

PART 3 HERE

\item{The Government of Skillistan wants society to have more P-types than it currently
does and decides to subsidize the tuition fee at Skillvard College.}
\begin{enumerate}
\item{Does this subsidy increase the number of P-types in society?}
\item{What are the aggregate consequences of this subsidy for residents and firms? In
other words, who benefits from the subsidy?}
\item{The Government is also a player in this economy. It cares most about the welfare
of residents, somewhat less about its own welfare and the least about the
welfare of firms. Is the fee subsidy a Pareto improvement over the status quo?
If not, suggest an alternative policy that would be a Pareto improvement.}
\end{enumerate}

PART 4 HERE

\end{enumerate}

\end{document}
